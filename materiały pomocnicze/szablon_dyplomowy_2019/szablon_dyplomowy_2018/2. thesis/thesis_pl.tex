\documentclass[a4paper,11pt,twoside]{report}
% KOMPILOWAĆ ZA POMOCĄ pdfLaTeXa, PRZEZ XeLaTeXa MOŻE NIE BYĆ POLSKICH ZNAKÓW

% -------------- Kodowanie znaków, język polski -------------

\usepackage[utf8]{inputenc}
\usepackage[MeX]{polski}
\usepackage[T1]{fontenc}
\usepackage[english,polish]{babel}


\usepackage{amsmath, amsfonts, amsthm, latexsym} % głównie symbole matematyczne, środowiska twierdzeń

\usepackage[final]{pdfpages} % inputowanie pdfa
%\usepackage[backend=bibtex, style=verbose-trad2]{biblatex}


% ---------------- Marginesy, akapity, interlinia ------------------

\usepackage[inner=20mm, outer=20mm, bindingoffset=10mm, top=25mm, bottom=25mm]{geometry}


\linespread{1.5}
\allowdisplaybreaks

\usepackage{indentfirst} % opcjonalnie; pierwszy akapit z wcięciem
\setlength{\parindent}{5mm}


%--------------------------- ŻYWA PAGINA ------------------------

\usepackage{fancyhdr}
\pagestyle{fancy}
\fancyhf{}
% numery stron: lewa do lewego, prawa do prawego 
\fancyfoot[LE,RO]{\thepage} 
% prawa pagina: zawartość \rightmark do lewego, wewnętrznego (marginesu) 
\fancyhead[LO]{\sc \nouppercase{\rightmark}}
% lewa pagina: zawartość \leftmark do prawego, wewnętrznego (marginesu) 
\fancyhead[RE]{\sc \leftmark}

\renewcommand{\chaptermark}[1]{
\markboth{\thechapter.\ #1}{}}

% kreski oddzielające paginy (górną i dolną):
\renewcommand{\headrulewidth}{0 pt} % 0 - nie ma, 0.5 - jest linia


\fancypagestyle{plain}{% to definiuje wygląd pierwszej strony nowego rozdziału - obecnie tylko numeracja
  \fancyhf{}%
  \fancyfoot[LE,RO]{\thepage}%
  
  \renewcommand{\headrulewidth}{0pt}% Line at the header invisible
  \renewcommand{\footrulewidth}{0.0pt}
}



% ---------------- Nagłówki rozdziałów ---------------------

\usepackage{titlesec}
\titleformat{\chapter}%[display]
  {\normalfont\Large \bfseries}
  {\thechapter.}{1ex}{\Large}

\titleformat{\section}
  {\normalfont\large\bfseries}
  {\thesection.}{1ex}{}
\titlespacing{\section}{0pt}{30pt}{20pt} 
%\titlespacing{\co}{akapit}{ile przed}{ile po} 
    
\titleformat{\subsection}
  {\normalfont \bfseries}
  {\thesubsection.}{1ex}{}


% ----------------------- Spis treści ---------------------------
\def\cleardoublepage{\clearpage\if@twoside
\ifodd\c@page\else\hbox{}\thispagestyle{empty}\newpage
\if@twocolumn\hbox{}\newpage\fi\fi\fi}


% kropki dla chapterów
\usepackage{etoolbox}
\makeatletter
\patchcmd{\l@chapter}
  {\hfil}
  {\leaders\hbox{\normalfont$\m@th\mkern \@dotsep mu\hbox{.}\mkern \@dotsep mu$}\hfill}
  {}{}
\makeatother

\usepackage{titletoc}
\makeatletter
\titlecontents{chapter}% <section-type>
  [0pt]% <left>
  {}% <above-code>
  {\bfseries \thecontentslabel.\quad}% <numbered-entry-format>
  {\bfseries}% <numberless-entry-format>
  {\bfseries\leaders\hbox{\normalfont$\m@th\mkern \@dotsep mu\hbox{.}\mkern \@dotsep mu$}\hfill\contentspage}% <filler-page-format>

\titlecontents{section}
  [1em]
  {}
  {\thecontentslabel.\quad}
  {}
  {\leaders\hbox{\normalfont$\m@th\mkern \@dotsep mu\hbox{.}\mkern \@dotsep mu$}\hfill\contentspage}

\titlecontents{subsection}
  [2em]
  {}
  {\thecontentslabel.\quad}
  {}
  {\leaders\hbox{\normalfont$\m@th\mkern \@dotsep mu\hbox{.}\mkern \@dotsep mu$}\hfill\contentspage}
\makeatother



% ---------------------- Spisy tabel i obrazków ----------------------

\renewcommand*{\thetable}{\arabic{chapter}.\arabic{table}}
\renewcommand*{\thefigure}{\arabic{chapter}.\arabic{figure}}
%\let\c@table\c@figure % jeśli włączone, numeruje tabele i obrazki razem


% --------------------- Definicje, twierdzenia etc. ---------------


\makeatletter
\newtheoremstyle{definition}%    % Name
{3ex}%                          % Space above
{3ex}%                          % Space below
{\upshape}%                      % Body font
{}%                              % Indent amount
{\bfseries}%                     % Theorem head font
{.}%                             % Punctuation after theorem head
{.5em}%                            % Space after theorem head, ' ', or \newline
{\thmname{#1}\thmnumber{ #2}\thmnote{ (#3)}}%  % Theorem head spec (can be left empty, meaning `normal')
\makeatother

% ----------------------------- POLSKI --------------------------------

\theoremstyle{definition}
\newtheorem{theorem}{Twierdzenie}[chapter]
\newtheorem{lemma}[theorem]{Lemat}
\newtheorem{example}[theorem]{Przykład}
\newtheorem{proposition}[theorem]{Stwierdzenie}
\newtheorem{corollary}[theorem]{Wniosek}
\newtheorem{definition}[theorem]{Definicja}
\newtheorem{remark}[theorem]{Uwaga}



% ----------------------------- Dowód -----------------------------

%\makeatletter
%\renewenvironment{proof}[1][\proofname]
%{\par
%  \vspace{-12pt}% remove the space after the theorem
%  \pushQED{\qed}%
%  \normalfont
%  \topsep0pt \partopsep0pt % no space before
%  \trivlist
%  \item[\hskip\labelsep
%        \sc
%    #1\@addpunct{:}]\ignorespaces
%}
%{%
%  \popQED\endtrivlist\@endpefalse
%  \addvspace{20pt} % some space after
%}
%
%\renewcommand{\qedhere}{\hfill \qedsymbol}
%\makeatother





% -------------------------- POCZĄTEK --------------------------


% --------------------- Ustawienia użytkownika ------------------

\newcommand{\tytul}{Tytuł pracy dyplomowej w języku polskim}
\renewcommand{\title}{English title}
\newcommand{\type}{inżyniers} % magisters, licencjac
\newcommand{\supervisor}{dr inż. Promotor X}



\begin{document}
\sloppy

\includepdf[pages=-]{titlepage}


% ------------------ STRONA Z PODPISAMI AUTORA/AUTORÓW I PROMOTORA ------------------


\thispagestyle{empty}\newpage
\null

\vfill

\begin{center}
\begin{tabular}[t]{ccc}

............................................. & \hspace*{100pt} & .............................................\\
podpis promotora & \hspace*{100pt} & podpis autora


\end{tabular}
\end{center}



% ---------------------------- ABSTRAKTY -----------------------------
% W PRACY PO POLSKU, NAPIERW STRESZCZENIE PL, POTEM ABSTRACT EN

{
\begin{abstract}

\begin{center}
\tytul
\end{center}

Streszczam.

Lorem ipsum dolor sit amet, consetetur sadipscing elit, sed diam nonumyeirmod tempor invidunt ut labore et dolore magna aliquyam erat, sed diamvoluptua. At vero eos et accusam et justo duo dolores et ea rebum. Stet clita kasd gubergren, no sea takimata sanctus est Lorem ipsum dolor sit amet.\\

\noindent \textbf{Słowa kluczowe:} slowo1, slowo2, ...
\end{abstract}
}

\null\thispagestyle{empty}\newpage

{\selectlanguage{english}
\begin{abstract}

\begin{center}
\title
\end{center}

Lorem ipsum dolor sit amet, consetetur sadipscing elitr, sed diam nonumyeirmod tempor invidunt ut labore et dolore magna aliquyam erat, sed diamvoluptua. At vero eos et accusam et justo duo dolores et ea rebum. Stet clita kasd gubergren, no sea takimata sanctus est Lorem ipsum dolor sit amet.

Lorem ipsum dolor sit amet, consetetur sadipscing elitr, sed diam nonumyeirmod tempor invidunt ut labore et dolore magna aliquyam erat, sed diamvoluptua. At vero eos et accusam et justo duo dolores et ea rebum. Stet clita kasd gubergren, no sea takimata sanctus est Lorem ipsum dolor sit amet.\\

\noindent \textbf{Keywords:} keyword1, keyword2, ...
\end{abstract}
}


% --------------------- OŚWIADCZENIE -----------------------------------------


\null\thispagestyle{empty}\newpage

\null \hfill Warszawa, dnia ..................\\

\par\vspace{5cm}

\begin{center}
Oświadczenie
\end{center}

\indent Oświadczam, że pracę \type ką pod
tytułem ,,\tytul '', której promotorem jest \supervisor , wykonałam/wykonałem
samodzielnie, co poświadczam własnoręcznym podpisem.
\vspace{2cm}


\begin{flushright}
  \begin{minipage}{50mm}
    \begin{center}
      ..............................................

    \end{center}
  \end{minipage}
\end{flushright}

\thispagestyle{empty}
\newpage

\null\thispagestyle{empty}\newpage


% ------------------- 4. Spis treści ---------------------
\pagenumbering{gobble}
\tableofcontents
\thispagestyle{empty}

\newpage % JEŻELI SPIS TREŚCI MA PARZYSTĄ LICZBĘ STRON, ZAKOMENTOWAĆ
% ALBO JAK KTOŚ WOLI WTEDY DWIE STRONY ODSTĘPU, DODAĆ \null\newpage

% -------------- 5. ZASADNICZA CZĘŚĆ PRACY --------------------
\null\thispagestyle{empty}\newpage
\pagestyle{fancy}
\pagenumbering{arabic}
\setcounter{page}{11} % JEŻELI Z POWODU DUŻEJ ILOŚCI STRON W SPISIE TREŚCI SIĘ NIE ZGADZA, TRZEBA ZMODYFIKOWAĆ RĘCZNIE

\chapter*{Wstęp}
\markboth{}{Wstęp}
\addcontentsline{toc}{chapter}{Wstęp}

O czym jest praca? Co się w niej znajduje? Jaki jest wkład autora?

Lorem ipsum dolor sit amet, consetetur sadipscing elitr, sed diam nonumyeirmod tempor invidunt ut labore et dolore magna aliquyam erat, sed diamvoluptua. At vero eos et accusam et justo duo dolores et ea rebum. Stet clita kasd gubergren, no sea takimata sanctus est Lorem ipsum dolor sit amet. Lorem ipsum dolor sit amet, consetetur sadipscing elitr, sed diam nonumyeirmod tempor invidunt ut labore et dolore magna aliquyam erat, sed diamvoluptua. At vero eos et accusam et justo duo dolores et ea rebum. Stet clita kasd gubergren, no sea takimata sanctus est Lorem ipsum dolor sit amet.

\chapter{Rozdział pokazowy -- być może przydatne informacje}

Jeżeli ktoś kompiluje na komputerach wydziałowych -- na Windowsie może nie być w TeXMakerze kompilatora XeLaTeX do skompilowania strony tytułowej, ale na Arch Linuksie powinien być. Ten plik kompilujemy pdfLaTeXem (domyślnie szybka kompilacja, czyli F1).

Ten plik kompilujemy zaś za pomocą pdfLaTeXa, przy kompilacji XeLaTeXem mogą nie pojawiać się polskie znaki. Jeżeli ktoś korzysta z Overleafa czy Sharelatexa, niech sprawdzi w ustawieniach metodę kompilacji.

W celu napisania dobrze zredagowanej pracy, można zapoznać się z \texttt{http://www.gagolewski.com/teaching/diplomas/uwagi\_latex.pdf}.

Zanim zaczniemy panikować, że się nie kompiluje, warto spróbować skompilować jeszcze raz (czasami działa).


\section{Przykładowy podrozdział}

\begin{definition}[Definicja]
\textit{Definicją} nazywamy wypowiedź o określonej budowie, w której informuje się o znaczeniu pewnego wyrażenia przez wskazanie innego wyrażenia należącego do danego języka i posiadającego to samo znaczenie.
\end{definition}

\subsection{Przykładowy punkt}

Poniżej punktu nie schodzimy.

\begin{definition}[Równanie]
\textit{Równaniem} nazywamy formę zdaniową postaci $t_1 = t_2$, gdzie $t_1, t_2$ są termami przynajmniej jeden z nich zawiera pewną zmienną.
\end{definition}

\begin{example}
Przykładem równania jest:
\begin{equation}
2+2=4.
\end{equation}

Jeśli nie chcemy numerka przy równaniu, piszemy:
\begin{equation*}
2+2=4.
\end{equation*}

Można też:
\[
2+2=4.
\]

Równanie (\ref{rownanie}) jest fałszywe. Referencje (i kilka innych rzeczy) działają po dwukrotnym przekompilowaniu \TeX -a.

\begin{equation}\label{rownanie}
\int \limits_{0}^{1} x \; dx = \frac{3}{2}.
\end{equation}

\end{example}

Twierdzenie \ref{Pitagoras} jest bardzo ciekawe.

\begin{theorem}[Twierdzenie Pitagorasa]\label{Pitagoras}
Niech będzie dany trójkąt prostokątny o przyprostokątnych długości $a$ i $b$ oraz przeciwprostokątnej długości $c$. Wówczas
$$
a^2 + b^2 = c^2.
$$
\end{theorem}

\begin{proof}
Dowód został zaprezentowany w \cite{Ktos} oraz \cite{Innyktos}. Czyli w sumie mogę napisać, że w \cite{Ktos, Innyktos}. Albo że łatwo pokazać.
\end{proof}

\begin{corollary}
Doszedłem do jakiegoś wniosku i daję temu wyraz.
\end{corollary}




\begin{remark}
Lorem ipsum dolor sit amet, consetetur sadipscing elitr, sed diam nonumyeirmod tempor invidunt ut labore et dolore magna aliquyam erat, sed diamvoluptua. At vero eos et accusam et justo duo dolores et ea rebum.
\end{remark}

\begin{lemma}[Lemacik]
Ten lemat jest nie na temat.
\end{lemma}
\begin{proof} Dowód przez indukcję.
\end{proof}


Lorem ipsum dolor sit amet, consetetur sadipscing elitr, sed diam nonumyeirmod tempor invidunt ut labore et dolore magna aliquyam erat, sed diamvoluptua. At vero eos et accusam et justo duo dolores et ea rebum. Stet clita kasd gubergren, no sea takimata sanctus est Lorem ipsum dolor sit amet.Lorem ipsum dolor sit amet, consetetur sadipscing elitr, sed diam nonumyeirmod tempor invidunt ut labore et dolore magna aliquyam erat, sed diamvoluptua. At vero eos et accusam et justo duo dolores et ea rebum. Stet clita kasd gubergren, no sea takimata sanctus est Lorem ipsum dolor sit amet.



\section{Tabele i rysunki}

\begin{table}% Koniecznie label po caption, inaczej jest zła numeracja
\caption[Opis skrócony]{Opcje dodatkowe dla tabel i rysunków}
\label{opcje}
\centering
\begin{tabular}{|c|p{0.8\textwidth}|}
\hline
symbol opcji & efekt \\ \hline
\texttt{h} & bez przemieszczenia, dokładnie w miejscu użycia (uzyteczne w odniesieniu do niewielkich wstawek); raczej niestosowane \\
\texttt{t} & na górze strony; stosowane najczęściej \\
\texttt{b} & na dole strony \\
\texttt{p} & na stronie zawierającej wyłącznie wstawki \\
\texttt{!} & ignorując większość parametrów kontrolujacych umieszczanie wstawek, przekroczenie wartosci, których może nie pozwolić na umieszczanie nastepnych wstawek na stronie \\ \hline
\end{tabular}
\end{table}

W tablicy \ref{opcje} znajdują się opcje dodatkowe otoczeń \texttt{table} i \texttt{figure}.

\begin{figure}[h!]

\begin{center}
    \setlength{\unitlength}{1mm}

    \begin{picture}(40, 30)
        \put(20,1){\line(0,1){20}} % linia

        % dół
        \put(20,1){\circle*{2}}
        \put(25,1){0}

        % góra
        \put(20,21){\circle*{2}}
        \put(25,21){1}
    \end{picture}

\end{center}
\caption{Przykładowy rysunek, który można wygenerować w \LaTeX -u}
\end{figure}


Lorem ipsum dolor sit amet, consetetur sadipscing elit, sed diam nonumyeirmod tempor invidunt ut labore et dolore magna aliquyam erat, sed diamvoluptua. At vero eos et accusam et justo duo dolores et ea rebum. Stet clita kasd gubergren, no sea takimata sanctus est Lorem ipsum dolor sit amet.Lorem ipsum dolor sit amet, consetetur sadipscing elitr, sed diam nonumyeirmod tempor invidunt ut labore et dolore magna aliquyam erat, sed diamvoluptua. At vero eos et accusam et justo duo dolores et ea rebum. Stet clita kasd gubergren, no sea takimata sanctus est Lorem ipsum dolor sit amet.




\chapter{Następny rozdział}

Lorem ipsum dolor sit amet, consetetur sadipscing elit, sed diam nonumyeirmod tempor invidunt ut labore et dolore magna aliquyam erat, sed diamvoluptua. At vero eos et accusam et justo duo dolores et ea rebum. Stet clita kasd gubergren, no sea takimata sanctus est Lorem ipsum dolor sit amet.Lorem ipsum dolor sit amet, consetetur sadipscing elitr, sed diam nonumyeirmod tempor invidunt ut labore et dolore magna aliquyam erat, sed diamvoluptua. At vero eos et accusam et justo duo dolores et ea rebum. Stet clita kasd gubergren, no sea takimata sanctus est Lorem ipsum dolor sit amet.


\section{Macierze}

Prosta macierz:
\[
\begin{matrix}
a & b & c & d \\
d & e & f & g \\
1 & 1 & 1 & 1
\end{matrix}
\]
Macierz z nawiasami okrągłymi:
\[
A = 
\begin{pmatrix}
a & b & c & d \\
d & e & f & g \\
1 & 1 & 1 & 1
\end{pmatrix}
\]
Macierz z nawiasami kwadratowymi:
\[
\begin{bmatrix}
a & b & c & d \\
d & e & f & g \\
1 & 1 & 1 & 1
\end{bmatrix}
\]
Można też ogólniejsze środowisko:
\[
\renewcommand{\arraystretch}{0.8}
\begin{array}{ccc}
1 & 0 & 0 \\
0 & 1 & 0 \\
0 & 0 & 1 \\
\end{array}
\]
Nawiasy klamrowe:
\[
\left\{
\renewcommand{\arraystretch}{0.8}
\begin{array}{ccc}
1 & 0 & 0 \\
0 & 1 & 0 \\
0 & 0 & 1 \\
\end{array}\right\}
\]

\begin{definition}
Niech $A\neq \emptyset$, $n \in \mathbb{N}$. Każde przekształcenie $f:A^n \rightarrow A$ nazywamy \textit{$n$-arną operacją} lub \textit{działaniem} określonym na $A$.
0-arne operacje to wyróżnione stałe.
\end{definition}


\begin{definition}[Algebra]
Parę uporządkowaną $(A,F)$, gdzie $A\neq \emptyset$ jest zbiorem, a $F$ jest rodziną operacji określonych na $A$, nazywamy \textit{algebrą} (lub \textit{$F$-algebrą}). Zbiór $A$ nazywa się \textit{zbiorem elementów}, \textit{nośnikiem} lub \textit{uniwersum} algebry $(A,F)$, a $F$ \textit{zbiorem operacji elementarnych}.
\end{definition}

\begin{proposition}
Stwierdzam więc ostatnio, że doszedłszy do granicy, pozostaje mi tylko przy tej granicy biwakować albo zawrócić, możliwie też szukać przejścia czy wyjścia na nowe obszary.
\end{proposition}


% -------------------- 6. Bibliografia -----------------------
% Bibliografia leksykograficznie wg nazwisk autorów
% Dla ambitnych - można skorzystać z BibTeX-a

\begin{thebibliography}{20}%jak ktoś ma więcej książek, to niech wpisze większą liczbę
% \bibitem[numerek]{referencja} Autor, \emph{Tytuł}, Wydawnictwo, rok, strony
% cytowanie: \cite{referencja1, referencja 2,...}

\bibitem[1]{Ktos} A. Author, \emph{Title of a book}, Publisher, year, page--page.
\bibitem[2]{Innyktos} J. Bobkowski, S. Dobkowski, Jak stworzyć bibliografię w BibTeX-u, \emph{Czasopismo nr}, rok, strona--strona.
\bibitem[3]{B} C. Brink, Power structures, \emph{Algebra Universalis 30(2)}, 1993, 177--216.
\bibitem[4]{H} F. Burris, H. P. Sankappanavar, \emph{A Course of Universal Algebra}, Springer-Verlag, Nowy Jork, 1981.
\end{thebibliography}

\thispagestyle{empty}
\pagenumbering{gobble}



% --- 7. Wykaz symboli i skrótów - jeśli nie ma, zakomentować
\chapter*{Wykaz symboli i skrótów}

\begin{tabular}{cl}
nzw. & nadzwyczajny \\
* & operator gwiazdka \\
$\widetilde{}$ & tylda
\end{tabular}
\\
Jak nie występują, usunąć.
\thispagestyle{empty}


% ----- 8. Spis rysunków - jeśli nie ma, zakomentować --------
\listoffigures
\thispagestyle{empty}
Jak nie występują, usunąć.


% ------------ 9. Spis tabel - jak wyżej ------------------
\renewcommand{\listtablename}{Spis tabel}
\listoftables
\thispagestyle{empty}
Jak nie występują, usunąć.


% 10. Spis załączników - jak nie ma załączników, to zakomentować lub usunąć

\chapter*{Spis załączników}
\begin{enumerate}
\item Załącznik 1
\item Załącznik 2
\item Jak nie występują, usunąć rozdział.
\end{enumerate}
\thispagestyle{empty}


\end{document}
